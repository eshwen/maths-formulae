\chapter{Trigonometry}
\label{sec:trig}

Sine as a power series:

\begin{equation}
    \sin x \equiv \sum_{n = 1}^{\infty} \frac{(-1)^{n - 1} \ x^{2n - 1}}{(2n - 1)!} = x - \frac{x^3}{3!} + \frac{x^5}{5!} - \frac{x^7}{7!} + \frac{x^9}{9!} - \dots
    \label{eq:sin_power_series}
\end{equation}

Cosine as a power series:

\begin{equation}
    \cos x \equiv \sum_{n = 0}^{\infty} \frac{(-1)^n \ x^{2n}}{(2n)!} = x - \frac{x^2}{2!} + \frac{x^4}{4!} - \frac{x^6}{6!} + \frac{x^8}{8!} - \dots
    \label{eq:cos_power_series}
\end{equation}

Tangent:

\begin{equation}
    \tan x = \frac{\sin x}{\cos x}
    \label{eq:tan}
\end{equation}

Sine in complex exponentials:

\begin{equation}
    \sin \theta = \frac{1}{2i} (e^{i\theta} - e^{-i\theta})
    \label{eq:sin_complex_exp}
\end{equation}

Cosine in complex exponentials:

\begin{equation}
    \cos \theta = \frac{1}{2} (e^{i\theta} + e^{-i\theta})
    \label{eq:cos_complex_exp}
\end{equation}

Sine squared:

\begin{equation}
    \sin^2 x = 1 - \cos^2 x = \frac{1}{2} (1 - \cos 2x)
    \label{eq:sin_squared}
\end{equation}

Cosine squared:

\begin{equation}
    \cos^2 x = 1 - \sin^2 x = \frac{1}{2} (1 + \cos 2x)
    \label{eq:cos_squared}
\end{equation}

Sine and cosine squared:

\begin{equation}
    \sin^2 x + \cos^2 x = 1
    \label{eq:sin_cos_squared}
\end{equation}

Double angle formula (sine):

\begin{equation}
    \sin 2x = 2\sin x \cos x
    \label{eq:double_angle_sin}
\end{equation}

Double angle formula (cosine):

\begin{equation}
    \cos 2x = \cos^2 x - \sin^2 x = 2cos^2 x - 1 = 1 - 2\sin^2 x
    \label{eq:double_angle_cos}
\end{equation}

Double angle formula (tangent):

\begin{equation}
    \tan 2x = \frac{2\tan x}{1 - \tan^2 x} \ \text{where} \ x \neq 45^{\circ}, 135^{\circ}, 225^{\circ} \dots
    \label{eq:double_angle_tan}
\end{equation}

Multiple angle formulae:

\begin{equation}
    \begin{aligned}
        \sin(A + B) &= \sin A \cos B + \cos A \sin B \\
        \sin(A - B) &= \sin A \cos B - \cos A \sin B \\
        \cos(A + B) &= \cos A \cos B - \sin A \sin B \\
        \cos(A - B) &= \cos A \cos B + \sin A \sin B \\\\
        \tan(A + B) &= \frac{\tan A + \tan B}{1 - \tan A \tan B} \ \text{where} \ (A + B) \neq 90^{\circ}, 270^{\circ} \dots\\
        \tan(A - B) &= \frac{\tan A - \tan B}{1 + \tan A \tan B} \ \text{where} \ (A - B) \neq 90^{\circ}, 270^{\circ} \dots\\\\
        \sin A \sin B &= \frac{1}{2}[\cos(A - B) - \cos(A + B)] \\
        \cos A \cos B &= \frac{1}{2}[\cos(A - B) + \cos(A + B)] \\
        \sin A \cos B &= \frac{1}{2}[\sin(A + B) + \sin(A - B)] \\\\
        \sin A + \sin B &= 2\sin\frac{A + B}{2} \cos\frac{A - B}{2}\\
        \sin A - \sin B &= 2\cos\frac{A + B}{2} \sin\frac{A - B}{2}\\
        \cos A + \cos B &= 2\cos\frac{A + B}{2} \cos\frac{A - B}{2}\\
        \cos A - \cos B &= 2\sin\frac{A + B}{2} \sin\frac{B - A}{2}
    \end{aligned}
    \label{eq:multi_angle}
\end{equation}

Cosecant (inverse sine):

\begin{equation}
    \csc x \defeq \frac{1}{\sin x}
    \label{eq:cosec}
\end{equation}

Secant (inverse cosine):

\begin{equation}
    \sec x \defeq \frac{1}{\cos x}
    \label{eq:sec}
\end{equation}

Cotangent (inverse tangent):

\begin{equation}
    \cot x \defeq \frac{1}{\tan x} = \frac{\cos x}{\sin x}
    \label{eq:cot}
\end{equation}

Inverse trigonometric identities:

\begin{equation}
    \begin{aligned}
        \csc^2 x &= 1 + \cot^2 x\\
        \sec^2 x &= 1 + \tan^2 x
    \end{aligned}
    \label{eq:inverse_trig}
\end{equation}


\section{Conversions between degrees and radians}

Converting degrees to radians:

\begin{equation}
    x[^{\circ}] = x[\radian] \times \frac{180}{\pi}
\end{equation}

\noindent{}Converting radians to degrees:

\begin{equation}
    x[\radian] = x[^{\circ}] \times \frac{\pi}{180}
\end{equation}


\section{Small angle approximations}

Valid for $x < 10^{\circ}$, with $x$ expressed in radians. Uses the expansions from Eqs.~\ref{eq:sin_power_series}, ~\ref{eq:cos_power_series}.

\

\noindent{}For $x$, sine, and tangent:
\begin{equation}
    x \sim \sin x \sim \tan x
\end{equation}

\noindent{}For cosine:

\begin{equation}
    \begin{aligned}
        \cos x &\sim 1 - \frac{x^2}{2}\\
        \cos (ax) &\sim 1 - \frac{(ax)^2}{2}
    \end{aligned}
\end{equation}


\section{Examples}

\subsection{Solving equations with multiple angle formulae}

Take $3\cos x + 4\sin x = 1$ as an example. Let

\begin{equation*}
    \begin{aligned}
    3\cos x + 4\sin x &= R\cos (x - \alpha) \\
    &= R(\cos x \cos \alpha + \sin x \sin \alpha) \leftarrow \text{from Eq.~\ref{eq:multi_angle}}
    \end{aligned}
\end{equation*}

\noindent{}Then

\begin{equation*}
    \underline{3}\cos x + \uwave{4}\sin x = \underline{R\cos\alpha} \cos x + \uwave{R\sin\alpha} \sin x
\end{equation*}

\noindent{}So $R\cos\alpha = 3$ and $R\sin\alpha = 4$, therefore

\begin{equation*}
    \begin{aligned}
    \tan\alpha &= 4/3\\
    \therefore \alpha &= \arctan (4/3)\\
    \alpha &= 0.441 \radian = 53.1^{\circ}
    \end{aligned}
\end{equation*}

\noindent{}If we square the relations above, we get $R^2\cos^2\alpha = 9$ and $R^2\sin^2\alpha = 16$. We can now calculate $R$:

\begin{equation*}
    \begin{aligned}
    R^2\sin^2\alpha + R^2\cos^2\alpha &= 25\\
    R^2(\sin^2\alpha + \cos^2\alpha) &= 25\\
    R^2 &= 25\\
    R &= 5
    \end{aligned}
\end{equation*}

\noindent{}So

\begin{equation*}
    \begin{aligned}
        3\cos x + 4\sin x = 1 &= 5\cos (x - 0.441)\\
        5\cos(x - 0.441) &= 1\\
        \color{red} x &\color{red}= \arccos (1/5) + 0.441
    \end{aligned}
\end{equation*}


\subsection{Expanding multiple angles as single and double angles}

Take $\sin (3x)$ as an example:

\begin{equation*}
    \begin{aligned}
        \sin(3x) &\equiv \sin(2x + x)\\
        &= \sin(2x)\cos x + \cos(2x) \sin x \leftarrow \text{From Eq.~\ref{eq:multi_angle}}\\
        &= 2\sin x \cos^2 x + \cos(2x) \ \sin x\\
        &= 2\sin x \cos^2 x + (\cos^2 x - \sin^2 x)\sin x\\
        &= \sin x (3\cos^2 x - \sin^2 x)\\
        &= \sin x (4\cos^2 x - 1)
    \end{aligned}
\end{equation*}
